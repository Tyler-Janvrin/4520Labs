\documentclass[12pt,emtex]{article}
\pagestyle{plain}
\usepackage{amssymb,amsthm,latexsym}
\usepackage{xcolor}
\usepackage{times}
\usepackage[reqno]{amsmath}
\usepackage[mathscr]{eucal}
\usepackage{graphics}
\usepackage{multicol}
\usepackage{pifont}
\usepackage{calc}
\usepackage{epsfig}
\psfigdriver{dvips}

\setlength{\topmargin}{-0.5in}
\setlength{\oddsidemargin}{-0.0in}
\setlength{\textheight}{9in}
\setlength{\textwidth}{6.5in}

\begin{document}
\title{\vspace{-25mm} \bf Cryptography 
\\ Assignment 2}
\date{}
\maketitle

\vspace{-20mm}

\begin{center}
	{\bf Dr. Charlie Obimbo \hfill Due:  February 11th, 2025}
\end{center}

\hrule

\vspace{6mm}

\

\noindent
{\bf Name: Tyler Janvrin \hspace{20mm} \hfill 
    To be done  in \LaTeX
}

\hfill Assignment is out of 10



	
\section{One-Time Pad}


Recall that in class we demonstrated how using the One-time pad was secure. In effect we showed how, in an effort to try and decrypt the cipher-text:

\medskip

``\texttt{ANKYODKYUREPFJBYOJDSPLREYIUNOFDOIUERFPLUYTS}"

\

\noindent
One cryptanalyst came up with the decryption:
``\texttt{MR MUSTARD WITH THE CANDLESTICK IN THE HALL}"; 
\
while another: 
``\texttt{MISS SCARLET WITH THE KNIFE IN THE LIBRARY}.

\begin{enumerate}
 \item Demonstrate how, using the character as the atomic operand, given the cipher-text:
 
\texttt{JACAOYOABLCYOUPOYTBN}
 
 One may get:
 
 
\begin{enumerate}
\item \texttt{COMMANDER IN CHIEF} 	 \hfill [1 mark]

\bigskip

The first step is to calculate values for the ciphertext and plaintext.

\begin{table}[h]
  \begin{tabular}{|l|l|l|l|l|l|l|l|l|l|l|l|l|l|l|l|l|l|l|l|}
  \hline
  1 & 2  & 3  & 4  & 5  & 6  & 7  & 8 & 9  & 10 & 11 & 12 & 13 & 14 & 15 & 16 & 17 & 18 & 19 & 20 \\ \hline
  J & A  & C  & A  & O  & Y  & O  & A & B  & L  & C  & Y  & O  & U  & P  & O  & Y  & T  & B  & N  \\ \hline
  9 & 0  & 2  & 0  & 14 & 24 & 14 & 0 & 1  & 11 & 2  & 24 & 14 & 20 & 15 & 14 & 24 & 19 & 1  & 13 \\ \hline
  C & O  & M  & M  & A  & N  & D  & E & R  & I  & N  & C  & H  & I  & E  & F  &    &    &    &    \\ \hline
  2 & 14 & 12 & 12 & 0  & 13 & 3  & 4 & 17 & 8  & 13 & 2  & 7  & 8  & 4  & 5  &    &    &    &    \\ \hline
  \end{tabular}
  \end{table}

Since the one-time pad operates on the text like so: PLAINTEXT + PAD = CIPHERTEXT, we can find the desred plaintext by
subtracting the plaintext from the ciphertext: PAD = CIPHERTEXT - PLAINTEXT.

\begin{table}[h]
  \begin{tabular}{|l|l|l|l|l|l|l|l|l|l|l|l|l|l|l|l|l|l|l|l|}
  \hline
  9 & 0   & 2   & 0   & 14 & 24 & 14 & 0  & 1   & 11 & 2   & 24 & 14 & 20 & 15 & 14 &  &  &  &  \\ \hline
  M & I   & N   & U   & S  &    &    &    &     &    &     &    &    &    &    &    &  &  &  &  \\ \hline
  2 & 14  & 12  & 12  & 0  & 13 & 3  & 4  & 17  & 8  & 13  & 2  & 7  & 8  & 4  & 5  &  &  &  &  \\ \hline
  E & Q   & U   & A   & L  & S  &    &    &     &    &     &    &    &    &    &    &  &  &  &  \\ \hline
  7 & -14 & -10 & -12 & 14 & 11 & 11 & \# & -16 & 3  & -11 & 22 & 7  & 12 & 11 & 9  &  &  &  &  \\ \hline
  M & O   & D   & 2   & 5  &    &    &    &     &    &     &    &    &    &    &    &  &  &  &  \\ \hline
  7 & 12  & 16  & 14  & 14 & 11 & 11 & \# & 10  & 3  & 15  & 22 & 7  & 12 & 11 & 9  &  &  &  &  \\ \hline
  H & M   & Q   & O   & O  & L  & L  & W  & K   & D  & P   & W  & H  & M  & L  & J  &  &  &  &  \\ \hline
  \end{tabular}
  \end{table}

  I did the computations in Excel and copied them over to this document. The final result for the pad that gives this ciphertext is is: HMQOOLLWKDPWHMLJ

  I ignored spaces, which I'm pretty sure is correct. If I hadn't, I would have gotten a slightly different result.
  \pagebreak


\item \texttt{THE SERGEANT AT ARMS}		\hfill [1 mark]

The process for this one is exactly the same:

\begin{table}[h]
  \begin{tabular}{|l|l|l|l|l|l|l|l|l|l|l|l|l|l|l|l|l|}
  \hline
  J    & A  & C  & A   & O  & Y  & O  & A  & B & L  & C   & Y  & O  & U  & P  & O  & Y  \\ \hline
  9    & 0  & 2  & 0   & 14 & 24 & 14 & 0  & 1 & 11 & 2   & 24 & 14 & 20 & 15 & 14 & 24 \\ \hline
  T    & H  & E  & S   & E  & R  & G  & E  & A & N  & T   & A  & T  & A  & R  & M  & S  \\ \hline
  19   & 7  & 4  & 18  & 4  & 17 & 6  & 4  & 0 & 13 & 19  & 0  & 19 & 0  & 17 & 12 & 18 \\ \hline
       &    &    &     &    &    &    &    &   &    &     &    &    &    &    &    &    \\ \hline
  9    & 0  & 2  & 0   & 14 & 24 & 14 & 0  & 1 & 11 & 2   & 24 & 14 & 20 & 15 & 14 & 24 \\ \hline
  M    & I  & N  & U   & S  &    &    &    &   &    &     &    &    &    &    &    &    \\ \hline
  19   & 7  & 4  & 18  & 4  & 17 & 6  & 4  & 0 & 13 & 19  & 0  & 19 & 0  & 17 & 12 & 18 \\ \hline
  E    & Q  & U  & A   & L  & S  &    &    &   &    &     &    &    &    &    &    &    \\ \hline
  \#\# & -7 & -2 & -18 & 10 & 7  & 8  & \# & 1 & -2 & -17 & 24 & -5 & 20 & -2 & 2  & 6  \\ \hline
  M    & O  & D  & 2   & 5  &    &    &    &   &    &     &    &    &    &    &    &    \\ \hline
  16   & 19 & 24 & 8   & 10 & 7  & 8  & \# & 1 & 24 & 9   & 24 & 21 & 20 & 24 & 2  & 6  \\ \hline
  Q    & T  & Y  & I   & K  & H  & I  & W  & B & Y  & J   & Y  & V  & U  & Y  & C  & G  \\ \hline
  \end{tabular}
  \end{table}

The final result for the pad that will produce JACAOYOABLCYOUPOYTBN from THE SERGEANT AT ARMS is QTYIKHIWBYJYVUYCG


\end{enumerate}

\newpage

\

\vspace{-20mm}

\item Joseph sends Aisha a message using a One-time pad. He also sends David another message using the same key. You were able to get both messages, as:


\texttt{0809 0302 0607 1A17 1A08 1C07 141D}
	and
	 	 \hfill [2 marks]
	
\texttt{0005 1311 1911 1907 0D09 1B08 130B}

 
  If the atomic operand is the bit, decrypt both messages and find the potential key.

Consider that one of the phrases may be from the following list: 

\

{\footnotesize \begin{center}
\begin{tabular}{|c|c|}  \hline
\ \ \ \ \ GORGEOUS SUSAN \ \ \ \ \ 
&
SHE ADORES JOHN
\\ \hline
NICOTINE IS BAD
&
MARIJUANAS LEGAL
\\ \hline
JUSTINE TRUDEAU
&
\ \ \ \ \ FLOYD MAYWEATHER \ \ \ \ \ 
\\ \hline
ANGELINA JOLIE
&
EMBEZZLED FUNDS \\ \hline
NANETTE WORKMAN
&
ELIZABETH MAY
\\ \hline
GRANT US PEACE
&
WE'RE AWESOME \\  \hline
\end{tabular}
\end{center}
}

When you're using a binary pad, you're using XOR. You go PLAINTEXT xor PAD = CRYPTEXT. XOR inverts the bits in the pad based on the bits in the plaintext, so if you have the message, you can go PLAINTEXT xor CRYPTEXT = PAD to derive the pad. 

Since there were only 12 possibilites for the plaintext, I just brute-forced it, using:

https://www.dcode.fr/xor-cipher to do the xor calculations.

I computed all twelve possible values for the pad for each of the plaintexts (24 total), and then using each possible pad against each plaintext until I encountered a message that made sense (if the message hadn't made sense, it wouldn't have been possible to find the correct pad).

When I decrypted 0005 1311 1911 1907 0D09 1B08 130B using the pad generated from EMBEZZLED FUNDS xor 0809 0302 0607 1A17 1A08 1C07 141D, I got the phrase MARVELOUS!AZIR. That's clearly not random, so I think I must have found the pad. The pad I got is: 4D44 4147 5C5D 5652 5E28 5A52 5A59 (with possibly some extra characters at the end, because the lengths aren't exactly the same).

\newpage





\end{enumerate}

\newpage

\

\vspace{-20mm}

    
    \section{Number Theory \& Hill Cipher}
    
    \vspace{2mm}
    \begin{enumerate}
    
        \setcounter{enumi}{3}
        
\item Is $2^{82589933}-1$ prime?   \ \  Yes, it is. \hfill [0.5]

\

Why? \ \ It was discovered to be prime by GIMPS (Great Internet Mersenne Prime Search) in 2018. At that time, it was the largest known prime. Source: 

https://www.mersenne.org/primes/press/M82589933.html \hfill [0.5]

\item Use Euclid's Algorithm to find \texttt{gcd(422774, 1009)}.
\ \ \ 
       [No Partial Marks] \hfill  (1 Mark)
       
       \

      $gcd(422774, 1009)$ \\
      Use Euclid's algorithm: \\ 
      $422774 = 419 * 1009 + 3$ \\
      $1009 = 336 * 3 + 1$ \\

      Therefore, the GCD of 422774 and 1009 is 3.






  


    \item Find the inverse of \texttt{1009 (mod 422774)}. [No Partial Marks] \hfill     (1 Mark)
    
    To find this, we can use the Extended Euclidean Algorithm and the results from Q5.
    
    $1 = 1009 - 336 * 3$ \\
    $1 = 1009 - 336 * (422774 - 419 * 1009)$ \\
    $1 = 1009 - 336 * 422774 +  336 * 419 * 1009$ \\
    $1 = 1009 - 336 * 422774 +  140784 * 1009$ \\
    $1 = 140785 * 1009 - 336 * 422774$ \\
    $1 = 140785 * 1009 (mod\ (422774))$ \\

    Therefore, the inverse of 1009 (mod 422774) is 140785.


\

	 \item  Find all solutions (between 1 \& 265) to the equation $35x \equiv 15$ (mod 265).         \hfill     [1]
	 
      $35x = 15 (mod\ 265)$ \\
      $7x = 3 (mod\ 53)$ \\
      $38 * 7x = 38 * 3 (mod\ 53)$ 38 is the multiplicative inverse of 7 \\ 
      $266x = 114 (mod\ 53)$ \\
      $x = 8 (mod\ 53)$ \\
      $x = 8, 61, 114, 167, 220,  (mod\ 265)$
      


\newpage


\

\vspace{-25mm}
          \item   (Hill-Cipher) Bob sends Alice the following code, in which the Hill-Cipher has been used, modulo 31. The key matrix used is:
    
          \vspace{-6mm}
            
            \
            
             	\[ K = 
            \left[
            	\begin{array}{cccccccccc}
                    5	&	30	& 	23 \\
                    6	&	30	& 	20 \\
                    26	&	1	&	9 \\
            \end{array}
            \right]
            \hspace{3mm}
             \mbox{and The Ciphertext $A$ is:  }
             \medskip
            \] 
          
          \
          
          \vspace{-6mm}
            
         \[  
            \left[
            \begin{array}{ccccccccccccccc}
T	&	1	&	H	&	I	&	C	&	O	&	Z	&	F	\\
F	&	W	&	B	&	T	&	S	&	P	&	B	&	J	\\
M	&	R	&	2	&	A	&	J	&	X	&	K	&	U	\\
           \end{array}
            \right]
            \]
            
                \medskip
           
          \hspace{-10mm}   
             If Bob used the following decimal encoding:
          
          \hspace{-10mm}   
                \begin{tabular}{|l||c|c|c|c|c|c|c|c|c|c|c|c|c|c|c|c|c|} \hline
                  % after \\: \hline or \cline{col1-col2} \cline{col3-col4} ...
                  Letter  & \hspace{0.5mm}  A   \hspace{0.5mm} & \hspace{0.5mm} B \hspace{0.5mm} & \hspace{0.5mm} C \hspace{0.5mm}
                          & \hspace{0.5mm}  D   \hspace{0.5mm} & \hspace{0.5mm} E \hspace{0.5mm} & \hspace{0.5mm} F \hspace{0.5mm}
                          & \hspace{0.5mm}  G   \hspace{0.5mm} & \hspace{0.5mm} H \hspace{0.5mm} & \hspace{0.5mm} I \hspace{0.5mm}
                          & \hspace{0.5mm}  J   \hspace{0.5mm} & \hspace{0.5mm} K \hspace{0.5mm} & \hspace{0.5mm} L \hspace{0.5mm}
                          & \hspace{0.5mm}  M   \hspace{0.5mm} &  \hspace{0.5mm} N  \hspace{0.5mm}   &  \hspace{0.5mm}   O  \hspace{0.5mm} &  \hspace{0.5mm} P  \hspace{0.5mm} \\ \hline
                  Code  & 0   &   1   &   2   &   3   &   4   &   5   &   6   &   7   &   8   &   9   &   10  &   11  &   12 & 13  &   14 & 15  \\ \hline \hline
                  Letter     &   Q   &   R   &   S   &   T   &   U   &   V   &   W   &   X   &   Y   &   Z & 0 & 1& 2& 3 & 4 &  \\ \hline
                  Code    &   16  &   17  &   18  &   19  &   20  &   21  &   22  &   23  &   24  &   25 & 26 & 27 & 28 & 29 & 30 & \\ \hline
                \end{tabular}
    
            
             \medskip
        
            (a)	Compute the inverse of the matrix $K$ (mod 31).					\hfill     [1]

            $
            \left[
            	\begin{array}{ccc|ccc}
                    5	&	30	& 23 & 1 & 0 & 0 \\
                    6	&	30	& 20 & 0 & 1 & 0 \\
                    26 & 1 &	9 & 0 & 0 & 1 \\
                \end{array}
            \right]
            $

            R1 + R3, R2 + R3

            $
            \left[
            	\begin{array}{ccc|ccc}
                    0	&	0	& 1 & 1 & 0 & 1 \\
                    1	&	0	& -2 & 0 & 1 & 1 \\
                    26 & 1 &	9 & 0 & 0 & 1 \\
                \end{array}
            \right]
            $

            R2 + 2R1
            
            $
            \left[
            	\begin{array}{ccc|ccc}
                    0	&	0	& 1 & 1 & 0 & 1 \\
                    1	&	0	& 0 & 2 & 1 & 3 \\
                    26 & 1 &	9 & 0 & 0 & 1 \\
                \end{array}
            \right]
            $

            R3 + 5R@
            
            $
            \left[
            	\begin{array}{ccc|ccc}
                    0	&	0	& 1 & 1 & 0 & 1 \\
                    1	&	0	& 0 & 2 & 1 & 3 \\
                    0 & 1 &	9 & 10 & 5 & 16 \\
                \end{array}
            \right]
            $

            R3 - 9R1

            $
            \left[
            	\begin{array}{ccc|ccc}
                    0	&	0	& 1 & 1 & 0 & 1 \\
                    1	&	0	& 0 & 2 & 1 & 3 \\
                    0 & 1 &	0 & 1 & 5 & 7 \\
                \end{array}
            \right]
            $

            Rearrange:

            $
            \left[
            	\begin{array}{ccc|ccc}
                    1	&	0	& 0 & 2 & 1 & 3 \\
                    0 & 1 &	0 & 1 & 5 & 7 \\
                    0	&	0	& 1 & 1 & 0 & 1 \\
                \end{array}
            \right]
            $
            
                \pagebreak

            

            
      

        

        
            (b)	Find the plaintext M.	(Remember to remove the gibberish \& punctuate it correctly.)	
                						\hfill       [1]

            First, we convert the letters in the matrix to their corresponding numbers:
            
            $
            \left[
            \begin{array}{cccccccc}
              19	&	27 &	7  &	8  &	2	&	14	&	25	&	5	\\
              5 	&	22 &	1  &	19 &	18	&	15	&	1	&	9	\\
              12	&	17	&	28	&	0	&	9	&	23	&	10	&	20	\\
           \end{array}
            \right]
            $

            Then, we multiply this matrix by the inverse of K:
            $
            \left[
            	\begin{array}{ccc}
                    2 & 1 & 3 \\
                    1 & 5 & 7 \\
                    1 & 0 & 1 \\
                \end{array}
            \right]
            $

            Yielding:

            $
            \left[
            \begin{array}{cccccccc}
              79	&	127 &	99  &	35  &	49	&	112 &	81	&	79	\\
              128 &	256 &	208  &	103 &	155	&	250	&	100	&	190	\\
              31	&	44	&	35	&	8	&	11	&	37	&	35	&	25	\\
           \end{array}
            \right]
            $

            Taking the mod of this, we get:

            $
            \left[
            \begin{array}{cccccccc}
              17	&	3 &	6  &	4  &	18	&	19 &	19	&	17	\\
              4 &	8 &	22  &	10 &	0	&	2	&	7	&	4	\\
              0	&	13	&	4	&	8	&	11	&	6	&	4	&	25	\\
           \end{array}
            \right]
            $

            Now, finally, we swap the letters out for the numbers:

            $
            \left[
            \begin{array}{cccccccc}
              R	&	D &	G  &	E  &	S	&	T &	T	&	R	\\
              E &	I &	W  &	K &	A	&	C	&	H	&	E	\\
              A	&	N	&	E	&	  I	&	L	&	G	&	E	&	Z	\\
           \end{array}
            \right]
            $

            That reads: Reading week is the rez.


\item  $-113$ (mod 10) \ = 7 \hfill [1]

      Proof: $-113 = -12 * 10 + 7$
            
    
    		 
    	\end{enumerate}
    


\end{document}

