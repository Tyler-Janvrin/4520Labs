\documentclass[12pt,emtex]{article}
\pagestyle{plain}
\usepackage{amssymb,amsthm,latexsym}
\usepackage{xcolor}
\usepackage{times}
\usepackage[reqno]{amsmath}
\usepackage[mathscr]{eucal}
\usepackage{graphics}
\usepackage{multicol}
\usepackage{pifont}
\usepackage{calc}
\usepackage{epsfig}
\psfigdriver{dvips}

\setlength{\topmargin}{-0.5in}
\setlength{\oddsidemargin}{-0.0in}
\setlength{\textheight}{9in}
\setlength{\textwidth}{6.5in}

\begin{document}
\title{\vspace{-25mm} \bf Cryptography 
\\ Assignment 2}
\date{}
\maketitle

\vspace{-20mm}

\begin{center}
	{\bf Dr. Charlie Obimbo \hfill Due:  February 11th, 2025}
\end{center}

\hrule

\vspace{6mm}

\

\noindent
{\bf Name: $\_\_\_\_\_\_\_\_\_\_\_\_\_\_\_\_\_\_\_\_\_\_\_\_\_\_\_\_\_\_\_\_\_\_\_\_\_\_\_\_\_\_$ \hspace{20mm} \hfill 
    To be done  in \LaTeX
}

\hfill Assignment is out of 10



	
\section{One-Time Pad}


Recall that in class we demonstrated how using the One-time pad was secure. In effect we showed how, in an effort to try and decrypt the cipher-text:

\medskip

``\texttt{ANKYODKYUREPFJBYOJDSPLREYIUNOFDOIUERFPLUYTS}"

\

\noindent
One cryptanalyst came up with the decryption:
``\texttt{MR MUSTARD WITH THE CANDLESTICK IN THE HALL}"; 
\
while another: 
``\texttt{MISS SCARLET WITH THE KNIFE IN THE LIBRARY}.

\begin{enumerate}
 \item Demonstrate how, using the character as the atomic operand, given the cipher-text:
 
\texttt{JACAOYOABLCYOUPOYTBN}
 
 One may get:
 
 
\begin{enumerate}
\item \texttt{COMMANDER IN CHIEF} 	 \hfill [1 mark]

\bigskip

\vspace{8mm}

\hrule \vspace{8mm}   \hrule \vspace{8mm}    
\hrule \vspace{8mm}   \hrule \vspace{8mm}   \hrule 
\

\item \texttt{THE SERGEANT AT ARMS}		\hfill [1 mark]

\vspace{12mm}

\hrule \vspace{8mm}   \hrule \vspace{8mm}    
\hrule \vspace{8mm}   \hrule \vspace{8mm}   \hrule 



\end{enumerate}

\newpage

\

\vspace{-20mm}

\item Joseph sends Aisha a message using a One-time pad. He also sends David another message using the same key. You were able to get both messages, as:


\texttt{0809 0302 0607 1A17 1A08 1C07 141D}
	and
	 	 \hfill [2 marks]
	
\texttt{0005 1311 1911 1907 0D09 1B08 130B}

 
  If the atomic operand is the bit, decrypt both messages and find the potential key.

Consider that one of the phrases may be from the following list: 

\

{\footnotesize \begin{center}
\begin{tabular}{|c|c|}  \hline
\ \ \ \ \ GORGEOUS SUSAN \ \ \ \ \ 
&
SHE ADORES JOHN
\\ \hline
NICOTINE IS BAD
&
MARIJUANAS LEGAL
\\ \hline
JUSTINE TRUDEAU
&
\ \ \ \ \ FLOYD MAYWEATHER \ \ \ \ \ 
\\ \hline
ANGELINA JOLIE
&
EMBEZZLED FUNDS \\ \hline
NANETTE WORKMAN
&
ELIZABETH MAY
\\ \hline
GRANT US PEACE
&
WE'RE AWESOME \\  \hline
\end{tabular}
\end{center}
}


\


\

\hrule \vspace{8mm}   \hrule \vspace{8mm}    
\hrule \vspace{8mm}   \hrule \vspace{8mm}
\hrule \vspace{8mm}   \hrule \vspace{8mm}    
\hrule \vspace{8mm}   \hrule \vspace{8mm}
\hrule \vspace{8mm}   \hrule \vspace{8mm}    
\hrule \vspace{8mm}   \hrule \vspace{8mm}
\hrule \vspace{8mm}   \hrule \vspace{8mm}
\hrule \vspace{8mm}   \hrule \vspace{8mm}    
\hrule \vspace{8mm}   \hrule \vspace{8mm}





\end{enumerate}

\newpage

\

\vspace{-20mm}

    
    \section{Number Theory \& Hill Cipher}
    
    \vspace{2mm}
    \begin{enumerate}
    
        \setcounter{enumi}{3}
        
\item Is $2^{82589933}-1$ prime?   \ \ \_\_\_\_\_\_\_\_\_\_\_\_\_\_\_\_\_\_\_\_\_\_\_\_	\hfill [0.5]

\

Why? \ \ \_\_\_\_\_\_\_\_\_\_\_\_\_\_\_\_\_\_\_\_\_\_\_\_\_\_\_\_\_\_\_\_\_\_\_\_\_\_\_\_\_\_\_\_\_\_\_\_ \hfill [0.5]

\item Use Euclid's Algorithm to find \texttt{gcd(422774, 1009)}.
\ \ \ 
       [No Partial Marks] \hfill  (1 Mark)


\

  \vspace{10mm}

    \hrule

    \vspace{10mm}

    \hrule

    \vspace{10mm}

    \hrule

    \vspace{10mm}

    \hrule

    \vspace{10mm}

  


    \item Find the inverse of \texttt{1009 (mod 422774)}. [No Partial Marks] \hfill     (1 Mark)



  
    \vspace{10mm}

    \hrule

    \vspace{10mm}

    \hrule

    \vspace{10mm}

    \hrule

    \vspace{10mm}

    \hrule


\

	 \item  Find all solutions (between 1 \& 265) to the equation $35x \equiv 15$ (mod 265).         \hfill     [1]

	    \vspace{8mm}  \hrule	    \vspace{8mm}  \hrule	    \vspace{8mm}  \hrule
	    

\newpage

\

\vspace{-25mm}
          \item   (Hill-Cipher) Bob sends Alice the following code, in which the Hill-Cipher has been used, modulo 31. The key matrix used is:
    
          \vspace{-6mm}
            
            \
            
             	\[ K = 
            \left[
            	\begin{array}{cccccccccc}
                    5	&	30	& 	23 \\
                    6	&	30	& 	20 \\
                    26	&	1	&	9 \\
            \end{array}
            \right]
            \hspace{3mm}
             \mbox{and The Ciphertext $A$ is:  }
             \medskip
            \] 
          
          \
          
          \vspace{-6mm}
            
         \[  
            \left[
            \begin{array}{ccccccccccccccc}
T	&	1	&	H	&	I	&	C	&	O	&	Z	&	F	\\
F	&	W	&	B	&	T	&	S	&	P	&	B	&	J	\\
M	&	R	&	2	&	A	&	J	&	X	&	K	&	U	\\
           \end{array}
            \right]
            \]
            
                \medskip
           
          \hspace{-10mm}   
             If Bob used the following decimal encoding:
          
          \hspace{-10mm}   
                \begin{tabular}{|l||c|c|c|c|c|c|c|c|c|c|c|c|c|c|c|c|c|} \hline
                  % after \\: \hline or \cline{col1-col2} \cline{col3-col4} ...
                  Letter  & \hspace{0.5mm}  A   \hspace{0.5mm} & \hspace{0.5mm} B \hspace{0.5mm} & \hspace{0.5mm} C \hspace{0.5mm}
                          & \hspace{0.5mm}  D   \hspace{0.5mm} & \hspace{0.5mm} E \hspace{0.5mm} & \hspace{0.5mm} F \hspace{0.5mm}
                          & \hspace{0.5mm}  G   \hspace{0.5mm} & \hspace{0.5mm} H \hspace{0.5mm} & \hspace{0.5mm} I \hspace{0.5mm}
                          & \hspace{0.5mm}  J   \hspace{0.5mm} & \hspace{0.5mm} K \hspace{0.5mm} & \hspace{0.5mm} L \hspace{0.5mm}
                          & \hspace{0.5mm}  M   \hspace{0.5mm} &  \hspace{0.5mm} N  \hspace{0.5mm}   &  \hspace{0.5mm}   O  \hspace{0.5mm} &  \hspace{0.5mm} P  \hspace{0.5mm} \\ \hline
                  Code  & 0   &   1   &   2   &   3   &   4   &   5   &   6   &   7   &   8   &   9   &   10  &   11  &   12 & 13  &   14 & 15  \\ \hline \hline
                  Letter     &   Q   &   R   &   S   &   T   &   U   &   V   &   W   &   X   &   Y   &   Z & 0 & 1& 2& 3 & 4 &  \\ \hline
                  Code    &   16  &   17  &   18  &   19  &   20  &   21  &   22  &   23  &   24  &   25 & 26 & 27 & 28 & 29 & 30 & \\ \hline
                \end{tabular}
    
            
             \medskip
        
            (a)	Compute the inverse of the matrix $K$ (mod 31).					\hfill     [1]

            
            
            
                \vspace{78mm}

        

        
            (b)	Find the plaintext M.	(Remember to remove the gibberish \& punctuate it correctly.)	
                						\hfill       [1]
            
            
                            \vspace{58mm}

\item  $-113$ (mod 10) \ = \ \ \ \_\_\_\_\_\_\_\_\_\_\_\_\_\_\_\_\_\_\_\_\_\_\_\_\_\_\_\_\_\_\_\_\_\_\_\_\_\_\_\_\_\_\_\_\_\_\_\_ \hfill [1]
              \medskip
            
    
    		 
    	\end{enumerate}
    


\end{document}

